\documentclass[a4paper,10pt]{letter}
\usepackage[margin=1in]{geometry}
\usepackage{lipsum}
\usepackage{xcolor}
\def\red#1{\textcolor{black}{#1}}


\begin{document}

% Sender Information
\begin{flushright}
    \begin{tabular}{l}
        \textbf{Piotr Stasiak} \\
        School of Mathematics, Statistics and Physics\\
        Newcastle University \\
        Newcastle  upon Tyne \\
        p.stasiak@newcastle.ac.uk \\
        \today
    \end{tabular}
\end{flushright}

\vspace{1cm}

% Recipient Information
The Editors,\\
Physical Review Letters\\
APS\\


\vspace{1cm}

Dear Editor,

\vspace{0.5cm}

We are pleased to submit our manuscript entitled 
\textbf{``Inverse energy transfer in three-dimensional quantum vortex flows''}
 to be considerated for publication in \textit{Physical Review Letters}. 
Our paper shows numerical evidence of the impact of superfluid vortex motion 
on the embedding \red{thermal excitations - the so-called normal
fluid}. Namely, a topological change in the superfluid vortex configuration 
(a \red{vortex} reconnection) triggers
a flux of normal fluid energy towards \red{large-scale motions}. 
This is surprising, as in \red{ordinary} fluids vortex 
reconnections generate an energy transfer in the opposite direction of
k-space, towards small-scale motions.

We clearly identify the underlying physical mechanism for this inverse 
energy transfer: the Kelvin waves 
generated on the superfluid vortices by the reconnection force 
helically the normal fluid, producing a
chiral flow \red{of thermal excitations}.  
\red{This helical imbalance is similar to what happens in 
\red{ordinary}
turbulence if it is artificially decimated by the numerics (as
demonstrated in Phys. Rev. Lett.
{\bf 108}, 164202, 2012).}
The important element to emphasize is that 
in superfluid \red{helium} this imbalance occurs naturally in the physical
world.

We believe that our findings will be of interest to readers of 
\textit{Physical Review Letters}, as they contribute to 
\red{a deeper understanding of turbulence};
%the lively, trans-disciplinary discussion regarding similiarities and 
%differences between turbulent flows in classical and quantum fluids;
\red{they will also} stimulate investigations
on the role of helicity in quantum turbulence, an aspect until now 
overlooked in the low-temperature community.  


Thank you for considering this manuscript; we look forward to your response.
\vspace{0.5cm}

On behalf of the authors,

\vspace{1cm}

\textbf{Piotr Stasiak}

\end{document}
