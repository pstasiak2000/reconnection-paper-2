% ****** Start of file apssamp.tex ******
%
%   This file is part of the APS files in the REVTeX 4.2 distribution.
%   Version 4.2a of REVTeX, December 2014
%
%   Copyright (c) 2014 The American Physical Society.
%
%   See the REVTeX 4 README file for restrictions and more information.
%
% TeX'ing this file requires that you have AMS-LaTeX 2.0 installed
% as well as the rest of the prerequisites for REVTeX 4.2
%
% See the REVTeX 4 README file
% It also requires running BibTeX. The commands are as follows:
%
%  1)  latex apssamp.tex
%  2)  bibtex apssamp
%  3)  latex apssamp.tex
%  4)  latex apssamp.tex
%
\documentclass[%
 reprint,
%superscriptaddress,
%groupedaddress,
%unsortedaddress,
%runinaddress,
%frontmatterverbose, 
%preprint,
%preprintnumbers,
%nofootinbib,
%nobibnotes,
%bibnotes,
 amsmath,amssymb,
 aps,
 prl,
%pra,
%prb,
%rmp,
%prstab,
%prstper,
%floatfix,
]{revtex4-2}

\usepackage{graphicx}% Include figure files
\usepackage{dcolumn}% Align table columns on decimal point
\usepackage{bm}% bold math
\usepackage{blindtext}
\usepackage{float}
\usepackage{caption}
\usepackage{cleveref}
\usepackage{subcaption}
\usepackage{xcolor}
%\usepackage{hyperref}% add hypertext capabilities
%\usepackage[mathlines]{lineno}% Enable numbering of text and display math
%\linenumbers\relax % Commence numbering lines

%\usepackage[showframe,%Uncomment any one of the following lines to test 
%%scale=0.7, marginratio={1:1, 2:3}, ignoreall,% default settings
%%text={7in,10in},centering,
%%margin=1.5in,
%%total={6.5in,8.75in}, top=1.2in, left=0.9in, includefoot,
%%height=10in,a5paper,hmargin={3cm,0.8in},
%]{geometry}


\newcommand{\etal}{{\it et al.}~}
\newcommand{\bom}{\boldsymbol{\omega}}

\def \s{\mathbf{s}}
\def \v{\mathbf{v}}
\def \x{\mathbf{x}}
\def \r{\mathbf{r}}
\def \k{\mathbf{k}}

\def \cm{\mathrm{cm}}
\def \cms{\mathrm{cm/s}}
\def \sec{\mathrm{s}}
\def \K{\mathrm{K}}

\def\red#1{\textcolor{red}{#1}}
\def\blue#1{\textcolor{blue}{#1}}
%%%%%%%%%%%%%%%%%%%%%%%%%%%%%%%%%



\begin{document}

\preprint{APS/123-QED}

\title{Emerging inverse energy transfer mechanism in coupled helium-4 vortex reconnections}

\author{P. Z. Stasiak}
\author{A. Baggaley}
\author{C.F. Barenghi}
\affiliation{School of Mathematics, Statistics and Physics, Newcastle University, Newcastle upon Tyne, NE1 7RU, United Kingdom}

\author{G. Krstulovic}
\affiliation{Universit\'e C\^ote d'Azur, Observatoire de la C\^ote d'Azur, CNRS,Laboratoire Lagrangre, Boulevard de l'Observatoire CS 34229 - F 06304 NICE Cedex 4, France}

\author{L. Galantucci}
\affiliation{Istituto per le Applicazioni del Calcolo ``M. Picone" IAC CNR, Via dei Taurini 19, 00185 Roma, Italy}

\date{\today}% It is always \today, today,
             %  but any date may be explicitly specified

\begin{abstract}
\blindtext
\end{abstract}

%\keywords{Suggested keywords}%Use showkeys class option if keyword
                              %display desired
\maketitle


\section{Introduction}





\begin{figure*}[t]
	\centering
	\begin{subfigure}[b]{0.24\textwidth}
		\centering
		\includegraphics*[width=\textwidth]{snap-1.png}
	\end{subfigure}
	\begin{subfigure}[b]{0.24\textwidth}
		\centering
		\includegraphics*[width=\textwidth]{snap-2.png}
	\end{subfigure}
    \begin{subfigure}[b]{0.24\textwidth}
		\centering
		\includegraphics*[width=\textwidth]{snap-3.png}
	\end{subfigure}
    \begin{subfigure}[b]{0.24\textwidth}
		\centering
		\includegraphics*[width=\textwidth]{snap-4.png}
	\end{subfigure}
    \hfill
	\caption{3D rendering of an orthogonal vortex configuration, undergoing a vortex reconnection. The red tube represents a superfluid vortex, where the radius has been greatly exaggerated for visual purposes, and the blue volume rendering represents the scaled normal fluid enstrophy $\bom^2/\bom^2_{max}$.}
\end{figure*}

\blue{
\blindtext[5]
}

\paragraph*{Main results.---} In this Letter, we use the Schwarz model to evolve vortex filaments $\s(\xi,t)$, where $\xi$ is the natural parametrisation of vortex lines, also known as the arclength. The normal fluid is coupled via the mutual friction force $\mathbf{f}_{ns}$ in a self-consistent manner using a recently developed technique in Ref. \cite{galantucciNewSelfconsistentApproach2020b}. Further details of the method are outlined in Ref. \cite{PunctuatedEnergyInjection} and the corresponding Supplementary Materials. 

\begin{figure}[H]
    \centering
    \includegraphics*[width=0.48\textwidth]{energy-spec.pdf}
    \caption{Normal fluid kinetic energy spectrum $E_k$ before reconnection (dashed lines), at reconnection (solid lines) and after reconnection (dotted lines) for $T=1.9K$ and $T=2.1K$. \emph{Inset:} Dissipation spectrum $D_k/\nu_n=k^2 E_k$ at the same snapshots in time.}
\end{figure}



\begin{figure}
    \centering
    \begin{subfigure}[b]{0.49\textwidth}
        \centering
        \includegraphics*[width=\textwidth]{inj-spec.pdf}
        \caption{}
    \end{subfigure}

    \begin{subfigure}[b]{0.49\textwidth}
    \centering
    \includegraphics*[width=\textwidth]{flux-spec.pdf}
    \caption{}
    \end{subfigure}
\hfill

\caption{\emph{Top:} Mutual friction injection spectrum $I_k$. \emph{Bottom:} Spectral normal fluid kinetic energy flux $\Pi_E=\int_k^{\infty}T(k)dk$, where $T(k)$ is the energy transfer function. \emph{Inset:} Post reconnection evolution of the integral length scale $\mathcal{L}_E=\pi/2\langle\mathbf{u}^2\rangle\int_0^{\infty}dk\,E_k/k$.}
\end{figure}




\begin{figure}
    \centering
    \includegraphics*[width=0.48\textwidth]{fmfDecompFig.pdf}
    \caption{Spectrum ratio of helical mutual friction modes $f^+(k)$ and $f^-(k)$ for $T=1.9K$ and $T=2.1K$.}
\end{figure}



\begin{figure}
    \centering
    \includegraphics*[width=0.48\textwidth]{hel-decomp.pdf}
    \caption{Balance of helical helicity modes $\mathcal{H}^+$ and $\mathcal{H}^-$. }
\end{figure}

\bibliography{references}% Produces the bibliography via BibTeX.

\end{document}
%
% ****** End of file apssamp.tex ******
