\documentclass[a4paper,10pt]{article}
\usepackage[margin=1in]{geometry}
\usepackage{cite}
\usepackage{amsmath}
\usepackage{graphicx}% Include figure files
\usepackage{bm}% bold math
\usepackage{float}
\usepackage{xcolor}
\def\red#1{\textcolor{black}{#1}}
\def\blue#1{\textcolor{blue}{#1}}

\def\refcomment#1{\textbf{\blue{\emph{#1}}}\\}
\hbadness=99999  % or any number >=10000

\def\red#1{\textcolor{red}{#1}}
\def\magenta#1{\textcolor{magenta}{#1}}


%%%%%%%%%%%%%%%%%%%%%%%%%%%%%%%%%
\newcommand*{\NOTE}[1]{\textbf{\color{red}[#1]}}
\newcommand*{\CHANGE}[1]{{\color{magenta}#1}}
\newcommand*{\DEL}[1]{{\color{green}#1}}




\begin{document}
\section*{\centering Point-by-point response to referee comments}

We thank the reviewers for taking the time to read over and give constructive feedback on our paper titled “\emph{Inverse energy transfer in three-dimensional quantum vortex flows}", submitted to PRL. Please see below the point-by-point response to the comments made by the reviewers.

\subsection*{Referee A}

\refcomment{For issue 1), authors
commented “We omitted the plot of the superfluid helicity from the
original manuscript,” but I did not understand where or which part was
omitted.}

\CHANGE{Originally we did not include the plot of the superfluid helicity
as we wanted our paper to focus on the properties of the normal fluid.
Following the reviewers' comments and for better
clarity, we  now include a sentence
which emphasizes that the total helicity (the sum of normal fluid's helicity
and superfluid's helicity) is not conserved.}\\

%Originally we wanted to include the plot of the superfluid helicity in the plot of the normal fluid helicity. We decided against this as in this paper we decide to only focus on the normal fluid aspect. \CHANGE{Nevertheless, we have included a sentence in the manuscript emphasizing that the total helicity of the flow, normal fluid plus superfluid, is nor conserved in the reconnection process.}\\   

\refcomment{For issue 2), the total sign of helicity for two
reconnections is neutral; one is negative and the other is positive.
Therefore, total symmetry breaking does not occur. This means that a
lot of reconnections do not cause total symmetry breaking or chiral
imbalance in quantum turbulence. As a result, the inverse cascade will
not happen.}

%We would like to highlight one subtlety of orthogonal reconnection simulations, which is that the two reconnections occur asynchronously. Given a large enough separation between vortices within the pair, as was this case for the configuration in our work, the large timescales required to observe a reconnection can lead to the pairs having very different reconnection times. Small deviations build up over time and the minimum distance $\delta$ between the vortices within each pair is not identical. As a result, the two reconnections do not occur at the same time. Pre-reconnection, it is true that the there is no global symmetry breaking, as the normal fluid helicity is not only negligible but is cancelled out globally. In fact, even at the reconnection of the first pair of vortices, the normal fluid helicity generated by the second pair does not change in light of the reconnection and remains negligible. This can be seen by computing the evolution of the normal fluid helicity in half-boxes containing exactly one pair of vortices. Therefore we do indeed have global symmetry breaking, even though we say that we are only considering one vortex pair. If we extended the plot of helicity in Fig.2a, one would see that the helicity injected is of the negative sign, with the global helicity returning back to 0. \\

\CHANGE{
To answer the Referee and generalise our findings, we have also
performed a study of the chiral imbalance generated by reconnections 
of Hopf-links (two linked vortex rings)
whose initial relative position is varied.
This allows to observe reconnections taking place at different 
angles. For Hopft-links, we observe the same strong helicity injection, 
increase of the 
integral scale, negative spectral fluxes and mutual friction force 
chirality  
that we have observed for open orthogonal vortex lines. 
These new results are described both in
the manuscript and in the Supplemental material.
Hence, we can claim with evidence that reconnections trigger inverse 
energy transfer. However, we stress that
this transfer of energy towards large scales may lead
to an inverse cascade only if reconnections are not exactly symmetrical, 
\textit{i.e.} (simplifying) if the number of reconnections of a given type 
is not exactly balanced by the same number of reconnections between vortices of opposite circulation. We suggest that this might be the case when the
flow's symmetry is broken by external forcing, as in thermal counterflow.}  \\

%further to make this point, in Fig.2 of the paper we include the reconnection data of Hopf-links that constitute a range of reconnection angles. The helicity evolution shows a strong injection of helicity and an increase in the integral length scale post reconnection. A similar analysis that we did on the helical decomposition also showed the same inverse energy transfer, but less stronger than for the orthogonal reconnection. We hope that this evidence is enough for the referee to see that we do indeed have symmetry breaking in the system. \\

%Furthermore, we would like to remind the referee that the effect that we are reporting o occurs locally, and our results do not attempt to suggest that this mechanism is responsible for inverse energy cascades in 3D quantum turbulence such as in counterflow channels. However, we would like to refer the referee to recent experimental results in counterflow channels such as in Ref~\cite{marakov2015,gao2017,mastracci2019,yui2020}. In these works, the experimental team often report the observation of large-scale normal fluid turbulence in thermal counterflow, which is still unexplained. In Ref.\cite{mastracci2019} the authors suggest that the large scale normal fluid turbulence may result from wake structures. Using data taken from a recent preprint \cite{galantucci2025} that three of us were part of, we calculated that the the flux of normal fluid due to wake structure is positive, implying an energy transfer to \emph{smaller} scales rather than larger scales, suggesting that wake structures are not likely to be the reason behind the large-scale normal fluid turbulence observed. \\

\refcomment{Regarding issue 3), a new normalization does not improve
the order of the kinetic energy spectrum and kinetic energy flux,
although Fig. 1 with dimensions in the response letter indeed shows a
reasonable order.}

\CHANGE{The new figures which we present in the revised manuscrit use
a temperature-independent normalization.
%In the revised manuscript, we have adapted the figures in the paper in order to use a normalization that is temperature independent scaling. %, which was the reason why the figures before showed an apparent inconsistency between the energy spectrum and the flux. 
The figures now show the same order, expected by the Referee,
as one would find when using dimensional units.} \\

\refcomment{The finding of inverse energy transfer by single reconnection is
interesting. However, other events cause forward energy transfer of
normal fluid, e.g., nonlinear interaction, wake propagation, energy
dissipations by mutual friction and viscosity, and so on. Therefore,
it is unclear whether the inverse energy cascade of normal fluid is
generated by reconnections in quantum turbulence. This stage is
premature to recommend to Physical Review Letters.}

\CHANGE{We think that by including the data regarding Hopf-link reconnections, we generalised our findings beyond a particular type of reconnection. This inverse energy transfer,
triggered by classical-like phenomena (chiral imbalance, which as in classical turbulence produces energy flowing towards large scales), is strikingly dissimilar from 
the direct energy transfer observed in vortex reconnections occurring in classical viscous fluids. Our work shows that vortex reconnections in quantum fluids may possess 
different characteristics from the ones taking place in classical fluids, depending on the fluid component we focus on: if we analyse superfluid vortices, the approach and 
separation is akin to classical dynamics (see our recent work, Stasiak et al., PNAS Vol. 122 No. 21 e2426064122 (2025)); if we observe the normal fluid we observe a non-classical 
energy transfer. Essentially, we have recognized a novel property of two-fluid
hysdrodynamics. Furthermore, our work adds additional elements to the lively discussion in the community regarding similarities and differences between classical and quantum 
turbulence. We think that together with the suggestion that reconnections may trigger an inverse energy cascade, these are the main findings of our investigation, which are of 
relevance for a broad audience such as PRL's readers. We have slightly changed the conclusions to emphasise these aspects.}

\subsection*{Referee B}

We thank the Referee for reading our revised version. We would like to insist, as it seems that is has not yet realized by the Referee, that our work is not about explaining the findings of Polanco and Krstulovic about the inverse cascade observed in that work for high counterflow. We have only mentioned that there might be a connection, but no claim is made. As a matter of fact, the comment is mild and we suggest that further studies are needed, so we open an new possible avenue of research. We have not mentioned such a claim in the abstract and not even in the introduction. 

The referee make a claim about head-on collision of quantum rings. 
One could perhaps find a specific configuration such that the helicity 
imbalance is weakened or even suppressed, however, our point is that this
is not necessarily the generic case and will not be relevant to experiments
of quantum turbulence. The Referee should notice that for orthogonal 
reconnection (discussed in the first version of our paper) the helicity 
for each pair is zero. In this newly revised version, we have included 
around 50 new reconnections of Hopf-links, which are topologically 
very different, as the initial helicity is non-zero. These simulations of 
Hopf-links reproduce what we have observed for orthogonal reconnections. 
We thus conclude, that our finding are robust and generic,
 and physically relevant.


%\begin{thebibliography}{4}
%    \bibitem{marakov2015}
%    A. Marakov, J. Gao, W. Guo, S.W. Van Sciver, G.G. Ihas, D.N. McKinsey, and W.F. Vinen, Visualization of the normal-fluid turbulence in counterflowing superfluid \textsuperscript{4}He, \emph{Phys. Rev. B} \textbf{91}, 094503 (2015).

%    \bibitem{gao2017}
%    J. Gao, E. Varga, W. Guo, and W.F. Vinen, Energy spectrum of thermal counterflow turbulence in superfluid helium-4, \emph{Phys. Rev. B} \textbf{96}, 094511 (2017)

%    \bibitem{mastracci2019}
%    B. Mastracci, S. Bao, W. Guo, and W.F. Vinen, Particle tracking velocimetry applied to thermal counterflow in superfluid \textsuperscript{4}He: motion of the normal fluid at small heat fluxes, \emph{Phys. Rev. Fluids} \textbf{4}, 083305 (2019).

%    \bibitem{yui2020}
%    S. Yui, H. Kobayashi, M. Tsubota, and W. Guo, Fully coupled dynamics of the two fluids in superfluid \textsuperscript{4}He: Anomalous anisotropic velocity fluctuations in counterflow, \emph{Phys. Rev. Lett.} \textbf{124}, 155301 (2020).

%    \bibitem{galantucci2025}
%    L. Galantucci, G. Krstulovic, and C. F. Barenghi, Quantum vortices leave a macroscopic signature in the normal fluid. \emph{arXiv preprint arXiv:2501.08309} (2025).


%\end{thebibliography}


\end{document}
