\documentclass[a4paper,10pt]{article}
\usepackage[margin=1in]{geometry}
\usepackage{cite}
\usepackage{amsmath}
\usepackage{graphicx}% Include figure files
\usepackage{bm}% bold math
\usepackage{float}
\usepackage{xcolor}
\def\red#1{\textcolor{black}{#1}}
\def\blue#1{\textcolor{blue}{#1}}

\def\refcomment#1{\textbf{\blue{\emph{#1}}}\\}
\hbadness=99999  % or any number >=10000


\begin{document}
\section*{\centering Point-by-point response to referee comments}

We thank the reviewers for taking the time to read over and give constructive feedback on our paper titled “\emph{Inverse energy transfer in three-dimensional quantum vortex flows}", submitted to PRL. Please see below the point-by-point response to the comments made by the reviewers.

\subsection*{Referee A}

\refcomment{For issue 1), authors
commented “We omitted the plot of the superfluid helicity from the
original manuscript,” but I did not understand where or which part was
omitted.}

\refcomment{For issue 2), the total sign of helicity for two
reconnections is neutral; one is negative and the other is positive.
Therefore, total symmetry breaking does not occur. This means that a
lot of reconnections do not cause total symmetry breaking or chiral
imbalance in quantum turbulence. As a result, the inverse cascade will
not happen.}

We would like to highlight one subtlety of orthogonal reconnection simulations, which is that the two reconnections occur asynchronously. Given a large enough separation between vortices within the pair, as was this case for the configuration in our work, the large timescales required to observe a reconnection can lead to the pairs having very different reconnection times. Small deviations build up over time and the minimum distance $\delta$ between the vortices within each pair is not identical. As a result, the two reconnections do not occur at the same time. Pre-reconnection, it is true that the there is no global symmetry breaking, as the normal fluid helicity is not only negligible but is cancelled out globally. In fact, even at the reconnection of the first pair of vortices, the normal fluid helicity generated by the second pair does not change in light of the reconnection and remains negligible. This can be seen by computing the evolution of the normal fluid helicity in half-boxes containing exactly one pair of vortices. Therefore we do indeed have global symmetry breaking, even though we say that we are only considering one vortex pair. If we extended the plot of helicity in Fig.2a, one would see that the helicity injected is of the negative sign, with the global helicity returning back to 0. \\

To further to make this point, in Fig.2 of the paper we include the reconnection data of Hopf-links that constitute a range of reconnection angles. The helicity evolution shows a strong injection of helicity and an increase in the integral length scale post reconnection. A similar analysis that we did on the helical decomposition also showed the same inverse energy transfer, but less stronger than for the orthogonal reconnection. We hope that this evidence is enough for the referee to see that we do indeed have symmetry breaking in the system. \\

Furthermore, we would like to remind the referee that the effect that we are reporting o occurs locally, and our results do not attempt to suggest that this mechanism is responsible for inverse energy cascades in 3D quantum turbulence such as in counterflow channels. However, we would like to refer the referee to recent experimental results in counterflow channels such as in Ref~\cite{marakov2015,gao2017,mastracci2019,yui2020}. In these work, the experimental team often report the observation of large-scale normal fluid turbulence in thermal counterflow, which is still unexplained. In Ref.\cite{mastracci2019} the authors suggest that the large scale normal fluid turbulence may result from wake structures. Using data taken from a recent preprint \cite{galantucci2025} that three of us were part of, we calculated that the the flux of normal fluid due to wake structure is positive, implying an energy transfer to \emph{smaller} scales rather than larger scales. \\

\refcomment{Regarding issue 3), a new normalization does not improve
the order of the kinetic energy spectrum and kinetic energy flux,
although Fig. 1 with dimensions in the response letter indeed shows a
reasonable order.}

In the revised manuscript, we have adapted the figures in the paper to use a normalization that is based on a temperature dependent scaling, which was the reason why the figures before showed an apparent inconsistency between the energy spectrum and the flux. The figures now show the same order as one would find when using dimensional units. 




\begin{thebibliography}{4}
    \bibitem{marakov2015}
    A. Marakov, J. Gao, W. Guo, S.W. Van Sciver, G.G. Ihas, D.N. McKinsey, and W.F. Vinen, Visualization of the normal-fluid turbulence in counterflowing superfluid \textsuperscript{4}He, \emph{Phys. Rev. B} \textbf{91}, 094503 (2015).

    \bibitem{gao2017}
    J. Gao, E. Varga, W. Guo, and W.F. Vinen, Energy spectrum of thermal counterflow turbulence in superfluid helium-4, \emph{Phys. Rev. B} \textbf{96}, 094511 (2017)

    \bibitem{mastracci2019}
    B. Mastracci, S. Bao, W. Guo, and W.F. Vinen, Particle tracking velocimetry applied to thermal counterflow in superfluid \textsuperscript{4}He: motion of the normal fluid at small heat fluxes, \emph{Phys. Rev. Fluids} \textbf{4}, 083305 (2019).

    \bibitem{yui2020}
    S. Yui, H. Kobayashi, M. Tsubota, and W. Guo, Fully coupled dynamics of the two fluids in superfluid \textsuperscript{4}He: Anomalous anisotropic velocity fluctuations in counterflow, \emph{Phys. Rev. Lett.} \textbf{124}, 155301 (2020).

    \bibitem{galantucci2025}
    L. Galantucci, G. Krstulovic, and C. F. Barenghi, Quantum vortices leave a macroscopic signature in the normal fluid. \emph{arXiv preprint arXiv:2501.08309} (2025).


\end{thebibliography}


\end{document}
