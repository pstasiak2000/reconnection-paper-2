\documentclass[a4paper,10pt]{article}
\usepackage[margin=1in]{geometry}
\usepackage{lipsum}
\usepackage{amsmath}
\usepackage{graphicx}% Include figure files
\usepackage{dcolumn}% Align table columns on decimal point
\usepackage{bm}% bold math
\usepackage{blindtext}
\usepackage{float}
\usepackage{caption}
\usepackage{cleveref}
\usepackage{subcaption}
\usepackage{xcolor}
\def\red#1{\textcolor{black}{#1}}
\def\blue#1{\textcolor{blue}{#1}}

\def\refcomment#1{\textbf{\blue{\emph{#1}}}\\}
\begin{document}
\section*{\centering Point-by-point response to referee comments}

We thank the reviewers for taking the time to read over and give constructive feedback on our paper titled “\emph{Inverse energy transfer in three-dimensional quantum vortex flows}", submitted to PRL. Please see below the point-by-point response to the comments made by the reviewers.

\subsection*{Referee A}


    \refcomment{1) p.1, right, "helicity, which is also an inviscid invariant": During
    the reconnection in a finite temperature, the total helicity including
    mutual friction force and normal fluid velocity is conserved? Normal
    fluid is handled as a viscous flow. Please describe to readers.}

    The presence of the mutual friction force introduces an \red{additional source of} dissipation into the normal fluid. By also computing the helicity in the superfluid component, we observe that during the reconnection event
the total helicity (sum of normal fluid helicity and superfluid helicity) is not conserved. We omitted the plot of the superfluid helicity
from the original manuscript so as to \red{avoid introducing}
the further concept 
of superfluid helicity to a paper whose narrative is \red{dedicated to the
normal fluid}. In the revised manuscript we have addressed the 
referee's concern by inserting (in page 4) an additional statement 
to clarify that the total helicity during the reconnection event is 
\textbf{not} conserved. \\

    \refcomment{2) p.3, right: Two pairs of initially orthogonal vortices are
    considered in the present configuration. This paper reports that one
    pair of vortices yield a negative helicity to the normal fluid
    velocity. Please mention whether the other pair of vortices yields the
    positive helicity or not. If both of them yield the negative helicity,
    the initial setup already causes a symmetry breaking.}

    In our manuscript, we indeed only describe the results referring to one pair of reconnecting 
    vortices. We confirm that the helicity injected by the other pair is indeed positive. 
    
%    The initial \red{vortex} configuration is set up such that the 
%superfluid velocity circulation in each half-box has the opposite sign to 
%the other half-box; \red{in this way, the superfluid velocity satisfies
%periodic boundary conditions over the entire computational 
%domain}.
%\red{It is important to appreciate that the two reconnection events do
%not occurr at the same time}. 
%In the timescale of the simulation \red{which we report in our manuscript}, 
%we only observe one pair of vortices which reconnects, as the reconnection 
%\red{of the other vortex pair occurs later, when the main impact of the first
%reconnection has vanished}.

%\red{We find that} during \red{this first reconnection}, the helicity contribution
%from the other vortex pair is negligible. \red{If we extend the
%simulation for a longer time and observe the second reconnection,
%we notice that} the second reconnection injects positive helicity 
%in the normal fluid, which then decays back to zero.

%Additionally, we computed the normal fluid helicity in each respective 
%half-box, confirming that the first reconnection injects negative
%helicity (computed only in its half-box), and, later, the second reconnection
%injects positive helicity (in its own half-box). 

%To \red{better} illustrate this point, we have modified the text to 
%explicity state that \red{the two} reconnections occur 
%asynchronously, \red{that the second reconnection is required by the
%periodic boundary conditions of our computational domain}, 
%\red{and that, physically, we focus the discussion of our results
%on the first reconnection only}.
%We have also added some extra details to Fig.2 in the \red{revised}
%manuscript which highlight this \red{feature} visually. \\


    \refcomment{3) Fig.3: In Fig.3(b), at $t = t_R$ the energy transfer at $T = 2.1K$ is
    higher than that at $T = 1.9K$. However, the energy at $T = 2.1K$ is lower
    than that at $T = 1.9K$. This is inconsistent. Please describe what
    happens. Is it better to normalize the $E(k)$ by $E_R$?}

    We appreciate the referee’s careful reading of the manuscript and 
their observation regarding a perceived inconsistency between the plots of 
the energy and energy flux spectra. The perceived inconsistency arises 
from the scalings employed. While the energy spectrum is in arbitrary 
units (in particular, in the non dimensional units \red{used in} the code), 
the energy flux is normalised with quantities which depend on the temperature. 
If, in fact, we plot the spectra in dimensional units (Figures 1 attached),
no inconsistency emerges. In the revised version of the code, following 
the referee's suggestion, we have scaled $E(k)$ by $E_R L_R$.
    
    \begin{figure}[h!]
    \centering
    \begin{subfigure}{0.49\textwidth}
        \centering
        \includegraphics*[width=\textwidth]{K_dimensional_clean.png}
        \caption{}
    \end{subfigure}
    \begin{subfigure}{0.49\textwidth}
        \centering
        \includegraphics*[width=\textwidth]{K_flux_dimensional_clean.png}
        \caption{}
    \end{subfigure}
    \caption{\emph{(a):} Normal fluid kinetic energy spectrum $E(k)$ before 
reconnection (dashed lines), at reconnection (solid lines) 
and after reconnection (dotted lines) for $T=1.9K$ (red)
and $T=2.1K$ (blue).\emph{(b):} Spectral normal fluid kinetic energy flux, $\Pi (k)$. 
}
\end{figure}
%Firstly, we would like to note that the two plots are scaled differently: The energy spectrum is displayed in units of the code, while the energy flux is scaled by the energy flux at reconnection. The discreplency that can be seen is a result of the scalings used. In the Fig.3b the values are scaled at the reconnection values, which are different at both the temperatures. In dimensional units, we have verified that in fact that both the flux and the energy spectrum are larger for $T=1.9K$ than $T=2.1K$. \\

    \refcomment{4) p.4, right, "This inverse energy transfer arises from the helical
    character of the friction generated by the Kelvin waves released by
    the reconnecting cusp": Is the inverse energy transfer due to the
    Kelvin wave? This reviewer thought that it is due to the mutual
    friction force being helical as described on the left of p.4.}

    The mutual friction force is helical as a result of the presence of Kelvin wave packets released along the vortices after the relaxation of the cusp generated at reconnection. The two elements, the helical character 
of the mutual friction force and the presence of Kelvin wave, are 
therefore inter-twinned. In the revised manuscript we have 
underlined this aspect  by slightly modifying the sentence pointed out.
by the referee.  
%    The referee is correct in thinking that the inverse energy transfer is due to the Kelvin waves. After the reconnection event, Kelvin waves are released by the reconnection cusp, which propagate outwards along both sides of the vortex. This in turn generates a helical profile in the normal fluid, however these Kelvin waves are damped at different rates. This in turn is what creates the imbalance of helicity.  
  
    \refcomment{5) Figure caption in Fig.1, $(t - t_0)/\tau_R$: $t_0$ is $t_R$ ?}

    The value $t_0$ should in fact be $t_R$, the typo has been corrected.\\
    
    \refcomment{6) Horizontal label in Fig.2: $(t - t_R)/\tau_R$ would be better than $(t- \tau_R)E_R^{1/2}/L_R$ ?}

    The authors agree with the comment of the referee that the current notation is not ideal. Instead, the figures have been changed to use the timescale in the horizontal axis of all time plots. Additionally to note, we have replaced the vertical axis label of Fig. 3b with $E_R \red{L_R}/\tau_R$, simply a change of symbols to make clear the non-dimensionalisation of units.\\
    
    \refcomment{7) Figure 3(a): Is there a power law for the slope of $E(k)$?}

    The figure has been updated with a power law of $k^{-3}$.\\
    
    \refcomment{8) p.4, left, "From Fig. 4 we determine the non-dimensional
    timescale...": Please describe the meaning of $t^*$.}

    The time $t^*$ was introduced as the time at which we observe a sufficient decay of the helical force ratio, where the difference is around 5\% between the positive and negative modes and remains relatively stable afterwards. In order to make this clearer, we have added a sentence in the text to explicity state this.

    \refcomment{9) p.4, left, "corresponding dimensionally to $\tau = 0.1$s for both
    temperatures.": Please describe where "$\tau = 0.1$s" comes from or what
    that means. Is the time when the helicity is changing?}
    
    The value of 0.1s comes from dimensionalising the values of $\tau=0.01$ for $T=1.9K$ and $\tau=0.005$ for $T=2.1K$, using the values of $\tilde{\tau}$ listed in the supplementary materials. \\
    
    \refcomment{10) Reference 53: B. ME should be M. E. Brachet. The year of 2026
    should be 2017.}

    The incorrect reference \red{noticed by the referee} has been
    corrected.\\




\newpage
\subsection*{Referee B}

\refcomment{The authors’ main motivation, as far as I understand, is to propose a mechanism for the inverse energy cascade which is numerically observed in counterflow superfluid turbulence [Polanco and Krstulovic, PRL (2020)].}

We want to clarify to the Referee that the purpose of our work is not 
at all to explain Polanco \& Krstulovic PRL 2020. Indeed, this
 article is only cited at the end of our manuscript as a possible physical 
setting where the new physical phenomena described in our manuscript 
could be relevant, and it could perhaps contribute to the origin of 
the inverse cascade reported in that paper (that one of us co-authors).  
To avoid possible confusion, we have reformulated the only paragraph 
in the conclusion where we cite Polanco\&Krstulovic.\\

\refcomment{As recalled by the authors, inverse energy cascades show up when helicity symmetry is explicitely broken by external forcing as in [Plunian et al., JFM (2020)]. Their claim, then, is that vortex reconnection events in superfluid turbulence will similarly break helicity symmetry and trigger, in this way, an inverse energy cascade in the flow.}

The referee \red{correctly} understood \red{that}
the most important claim of our paper concerns the sign-defined helicity 
injection. We would like to highlight that we have consciously avoided 
talking about \red{an} {\it inverse energy cascade}, as we are well aware 
that our system is not turbulent. We have used the words {\it inverse energy transfer} instead.\\

\refcomment{While the paper’s idea is interesting and worth pursuing, I believe the authors fail to provide even minimal evidence to support their case. Below, I address my main points.
(i) First, the use of the word “turbulence” in the very first line (and used 16 times in other parts) of the manuscript is misleading: the flow studied by the authors is just a collision of vortex filaments. There is no turbulent flow at all here. It is all about a specific problem on superfluid flow instability.}

As the Referee is certainly aware, vortex reconnections play an important role in turbulence.We have never claimed that our system is turbulent, but we do think that our findings have a strong impact on quantum turbulence. In the text, we have only referred to `turbulence' in the introduction to motivate our work and to explain its relevance to a broad audience, and at the very end to comment on its implications in the conclusions. Explaining cascades, turbulence and invariants is necessary to understand the role of helicity on transfer towards large scales. We still believe that the use of the word turbulence is 
appropriate in our manuscript \red{to put our results into the most 
relevant context}. 

\red{Nevertheless, to take the referee's comment into account,}
we have added a sentence in the introduction \red{(page 1)}
to remind the reader of the importance of vortex reconnections 
in quantum turbulent flows, and a comment in the conclusions 
about when our finding might be important for quantum turbulence. \\

\refcomment{(ii) The work of Polanco and Krstulovic [PRL (2020)] gives strong indication that counterflow superfluid turbulence behaves (for large enough counterflow velocities) as a quasi two-dimensional syste and, therefore, an inverse energy cascade is expected to occur. Polanco and Krstulovic have also observed, by the way, the energy spectrum decay of the direct enstrophy cascade, in further qualitative agreement with the Batchelor-Kraichnan picture of 2D turbulence.}

We do agree with the Referee’s understanding of Polanco\& Krstulovic main result. 
%However, we do not understand why the Referee mentions it as they have not formulated any question or critcism.\\
However, we would like again to point out that the purpose of our work is not at all to explain Polanco \& Krstulovic PRL 2020. Indeed, this article is only cited at the end of our manuscript, as a possible physical setting where the new physical phenomena described in our manuscript could be relevant. \red{The relevance of the large counterflow velocities in our mechanism for inverse energy transfer is in addition addressed in the next point.}
 


\refcomment{(iii) The helicity production for an individual vortex reconnection event can be, of course, positive or negative. However, it is unlikley that this will be the case for the global helicity production associated to the complex vortex tangle of counterflow turbulence. I note that the previous results of [Plunian et al., JFM (2020)] refer to flow regimes where helicity symmetry is broken across all scales by the external forcing.}

Indeed, the helicity injection produced by random vortex reconnections 
is not sign-defined, but the helicity injected by a single \red{reconnection}
is.  In isotropic turbulence, the helicity injection produced by vortex reconnections is certainly positive and negative \red{and the overall helicity injected
might cancel out. However,} in the case where the system is highly anisotropic and the vortex density is sufficiently large \red{(as it is the case when counterflow velocities are large, where vortices tend to lie in a plane perpendicular to the heat flow)}, such symmetry breaking induced by the counterflow velocity could favour an asymmetry in the reconnection events triggering the overall production of helicity of one given sign. For this reason, at the end of the conclusion, we mentioned counterflow turbulence as such an example and raised the question of whether the findings of Polanco\&Krstulovic could be ultimately related to the new mechanism reported in our Letter.  Providing a definitive answer to this open question is undoubtedly challenging and beyond the scope of our work. In the revised version, we have softened and reformulated the sentence referring to Polanco\&Krstulovic.\\

\refcomment{(iv) The vortex reconnection mechanism for the inverse energy cascade would imply that energy would flow from very small scales (around the atomic sizes of the vortex cores) towards the integral length scales. That is not what is observed, as discussed by Polanco and Krstulovic [PRL (2020)].}

We remind the Referee that Polanco\&Krstulovic used the HVBK model. 
This model does not describe \red{individual} quantised vortices 
\red{and their reconnections, but only a continuous density of vortex lines}, 
\red{thus providing information only at lenght scales 
much larger than the average vortex separation.} In the HVBK model
the energy injection
\red{occurs} via an external term whose acting scale can be chosen
arbitrarily. 
\red{In the HVBK model} the inverse cascade is triggered by the nonlinear 
term when the counterflow 
is strong enough, which leads to a strong anisotropy. 
In our coupled model, the normal fluid energy injection is produced 
by vortex reconnections, and it is indeed transferred towards large scales 
\red{by non-linear terms, at $k < 40$ (Fig. 3b), \textit{i.e.} at scales much larger than the vortex core (we recall that our box is
$1mm^3$)}. 
We remind the Referee that one needs to be careful \red{when}
comparing \red{the HVBK model} and our coupled model,
as they describe the physics at very different scales.  Again, by writing the sentence about Polanco\&Krstulovic, we simply raise a question that needs further investigation.\\

\refcomment{(v) Finally, the “inverse energy cascade” detected by the authors may be unrelated to helicity production. One could just as easily argue that the perturbation/production of the standard component of the fluid through vortex reconnection is initially localized around the point where the filaments come into close contact. Subsequently, this normal component perturbation is strained by the specific background velocity configuration of the surrounding normal fluid. Helicity would not play any relevant dynamical role here.}

The Referee raises a good point.  In a recent paper [arXiv:2501.08309], three of us studied the wakes generated by moving superfluid vortices (with respect to the normal fluid). In this case, however, energy results to be transferred to \textit{small} scales, discarding the scenario proposed by the Referee. \red{We attach here (Fig. 2) the plot of the energy flux spectrum corresponding to the development of the wakes: all values are positive, showing a direct energy cascade.} \\



%
\begin{figure}[h!]
    \centering
    \includegraphics*[width=0.48\textwidth]{flux_wakes.png}
    \caption{Spectral normal fluid kinetic energy flux, $\Pi (k)$ for the wakes generated in the normal fluid, data from [arXiv:2501.08309].}.
    
\end{figure}

\refcomment{For the above reasons, I do not recommend the paper for publication in the PRL.}

\noindent \red{We trust} that we have clarified to the Referee the goal and main results of our Letter, and at the same time, provided enough arguments to emphasize the impact of our Letter.







\end{document}
