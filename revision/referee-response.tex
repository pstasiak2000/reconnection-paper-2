\documentclass[a4paper,10pt]{letter}
\usepackage[margin=1in]{geometry}
\usepackage{lipsum}
\usepackage{xcolor}
\def\red#1{\textcolor{red}{#1}}


\begin{document}

\begin{enumerate}
    \item The total helicity during the reconnection event is not conserved. The presence of the mutual friction force introduces a dissipation into the normal fluid. Overall, by computing the helicity in the superfluid component, we confirmed this. We decided to omit this plot from the manuscript so as to not introduce further concepts of superfluid helicity to a paper whose narrative is normal fluid centric. We have addressed the referees concern by inserting an additional statement to clarify that the helicity during the reconnection event is not in fact conserved. 
    \item In the timescale of the simulation, we only observe one pair of the vortices which reconnect.  
    \item 
    \item 
    \item The value $t_0$ should in fact be $t_R$, the typo has been corrected.
    \item The authors agree with the comment of the referee that the current notation is not ideal. Instead, the figures have been changed to use the timescale in the horizontal axis of all time plots. Additionally to note, we have replaced the vertical axis label of Fig. 3b with $E_R/\tau_R$, simply a change of symbols to make clear the non-dimensionalisation of units.
    \item The figure has been updated with a power law of $k^{-3}$ 
    \item The time $t^*$ was introduced as the time at which we observe a sufficient decay of the helical force ratio, where the difference is around 5\% between the positive and negative modes and remains relatively stable afterwards. In order to make this clearer, we have added a sente nce in the text to explicity state this.
    \item The value of 0.1s comes from dimensionalising the values of $\tau=0.01$ for $T=1.9K$ and $\tau=0.005$ for $T=2.1K$, using the values of $\tilde{\tau}$ listed in the supplementary materials. 
    \item Incorrect reference corrected as noted by the reviewer.
\end{enumerate}
\end{document}
